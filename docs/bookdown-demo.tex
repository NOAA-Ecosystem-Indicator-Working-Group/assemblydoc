% Options for packages loaded elsewhere
\PassOptionsToPackage{unicode}{hyperref}
\PassOptionsToPackage{hyphens}{url}
%
\documentclass[
]{book}
\usepackage{amsmath,amssymb}
\usepackage{lmodern}
\usepackage{ifxetex,ifluatex}
\ifnum 0\ifxetex 1\fi\ifluatex 1\fi=0 % if pdftex
  \usepackage[T1]{fontenc}
  \usepackage[utf8]{inputenc}
  \usepackage{textcomp} % provide euro and other symbols
\else % if luatex or xetex
  \usepackage{unicode-math}
  \defaultfontfeatures{Scale=MatchLowercase}
  \defaultfontfeatures[\rmfamily]{Ligatures=TeX,Scale=1}
\fi
% Use upquote if available, for straight quotes in verbatim environments
\IfFileExists{upquote.sty}{\usepackage{upquote}}{}
\IfFileExists{microtype.sty}{% use microtype if available
  \usepackage[]{microtype}
  \UseMicrotypeSet[protrusion]{basicmath} % disable protrusion for tt fonts
}{}
\makeatletter
\@ifundefined{KOMAClassName}{% if non-KOMA class
  \IfFileExists{parskip.sty}{%
    \usepackage{parskip}
  }{% else
    \setlength{\parindent}{0pt}
    \setlength{\parskip}{6pt plus 2pt minus 1pt}}
}{% if KOMA class
  \KOMAoptions{parskip=half}}
\makeatother
\usepackage{xcolor}
\IfFileExists{xurl.sty}{\usepackage{xurl}}{} % add URL line breaks if available
\IfFileExists{bookmark.sty}{\usepackage{bookmark}}{\usepackage{hyperref}}
\hypersetup{
  pdftitle={Technical Indicator Assembly Document for NOAA NaMES},
  pdfauthor={Willem Klajbor, The NOAA Ecosystem Indicators Working Group},
  hidelinks,
  pdfcreator={LaTeX via pandoc}}
\urlstyle{same} % disable monospaced font for URLs
\usepackage{color}
\usepackage{fancyvrb}
\newcommand{\VerbBar}{|}
\newcommand{\VERB}{\Verb[commandchars=\\\{\}]}
\DefineVerbatimEnvironment{Highlighting}{Verbatim}{commandchars=\\\{\}}
% Add ',fontsize=\small' for more characters per line
\usepackage{framed}
\definecolor{shadecolor}{RGB}{248,248,248}
\newenvironment{Shaded}{\begin{snugshade}}{\end{snugshade}}
\newcommand{\AlertTok}[1]{\textcolor[rgb]{0.94,0.16,0.16}{#1}}
\newcommand{\AnnotationTok}[1]{\textcolor[rgb]{0.56,0.35,0.01}{\textbf{\textit{#1}}}}
\newcommand{\AttributeTok}[1]{\textcolor[rgb]{0.77,0.63,0.00}{#1}}
\newcommand{\BaseNTok}[1]{\textcolor[rgb]{0.00,0.00,0.81}{#1}}
\newcommand{\BuiltInTok}[1]{#1}
\newcommand{\CharTok}[1]{\textcolor[rgb]{0.31,0.60,0.02}{#1}}
\newcommand{\CommentTok}[1]{\textcolor[rgb]{0.56,0.35,0.01}{\textit{#1}}}
\newcommand{\CommentVarTok}[1]{\textcolor[rgb]{0.56,0.35,0.01}{\textbf{\textit{#1}}}}
\newcommand{\ConstantTok}[1]{\textcolor[rgb]{0.00,0.00,0.00}{#1}}
\newcommand{\ControlFlowTok}[1]{\textcolor[rgb]{0.13,0.29,0.53}{\textbf{#1}}}
\newcommand{\DataTypeTok}[1]{\textcolor[rgb]{0.13,0.29,0.53}{#1}}
\newcommand{\DecValTok}[1]{\textcolor[rgb]{0.00,0.00,0.81}{#1}}
\newcommand{\DocumentationTok}[1]{\textcolor[rgb]{0.56,0.35,0.01}{\textbf{\textit{#1}}}}
\newcommand{\ErrorTok}[1]{\textcolor[rgb]{0.64,0.00,0.00}{\textbf{#1}}}
\newcommand{\ExtensionTok}[1]{#1}
\newcommand{\FloatTok}[1]{\textcolor[rgb]{0.00,0.00,0.81}{#1}}
\newcommand{\FunctionTok}[1]{\textcolor[rgb]{0.00,0.00,0.00}{#1}}
\newcommand{\ImportTok}[1]{#1}
\newcommand{\InformationTok}[1]{\textcolor[rgb]{0.56,0.35,0.01}{\textbf{\textit{#1}}}}
\newcommand{\KeywordTok}[1]{\textcolor[rgb]{0.13,0.29,0.53}{\textbf{#1}}}
\newcommand{\NormalTok}[1]{#1}
\newcommand{\OperatorTok}[1]{\textcolor[rgb]{0.81,0.36,0.00}{\textbf{#1}}}
\newcommand{\OtherTok}[1]{\textcolor[rgb]{0.56,0.35,0.01}{#1}}
\newcommand{\PreprocessorTok}[1]{\textcolor[rgb]{0.56,0.35,0.01}{\textit{#1}}}
\newcommand{\RegionMarkerTok}[1]{#1}
\newcommand{\SpecialCharTok}[1]{\textcolor[rgb]{0.00,0.00,0.00}{#1}}
\newcommand{\SpecialStringTok}[1]{\textcolor[rgb]{0.31,0.60,0.02}{#1}}
\newcommand{\StringTok}[1]{\textcolor[rgb]{0.31,0.60,0.02}{#1}}
\newcommand{\VariableTok}[1]{\textcolor[rgb]{0.00,0.00,0.00}{#1}}
\newcommand{\VerbatimStringTok}[1]{\textcolor[rgb]{0.31,0.60,0.02}{#1}}
\newcommand{\WarningTok}[1]{\textcolor[rgb]{0.56,0.35,0.01}{\textbf{\textit{#1}}}}
\usepackage{longtable,booktabs,array}
\usepackage{calc} % for calculating minipage widths
% Correct order of tables after \paragraph or \subparagraph
\usepackage{etoolbox}
\makeatletter
\patchcmd\longtable{\par}{\if@noskipsec\mbox{}\fi\par}{}{}
\makeatother
% Allow footnotes in longtable head/foot
\IfFileExists{footnotehyper.sty}{\usepackage{footnotehyper}}{\usepackage{footnote}}
\makesavenoteenv{longtable}
\usepackage{graphicx}
\makeatletter
\def\maxwidth{\ifdim\Gin@nat@width>\linewidth\linewidth\else\Gin@nat@width\fi}
\def\maxheight{\ifdim\Gin@nat@height>\textheight\textheight\else\Gin@nat@height\fi}
\makeatother
% Scale images if necessary, so that they will not overflow the page
% margins by default, and it is still possible to overwrite the defaults
% using explicit options in \includegraphics[width, height, ...]{}
\setkeys{Gin}{width=\maxwidth,height=\maxheight,keepaspectratio}
% Set default figure placement to htbp
\makeatletter
\def\fps@figure{htbp}
\makeatother
\setlength{\emergencystretch}{3em} % prevent overfull lines
\providecommand{\tightlist}{%
  \setlength{\itemsep}{0pt}\setlength{\parskip}{0pt}}
\setcounter{secnumdepth}{5}
\usepackage{booktabs}
\usepackage{amsthm}
\makeatletter
\def\thm@space@setup{%
  \thm@preskip=8pt plus 2pt minus 4pt
  \thm@postskip=\thm@preskip
}
\makeatother
\ifluatex
  \usepackage{selnolig}  % disable illegal ligatures
\fi
\usepackage[]{natbib}
\bibliographystyle{apalike}

\title{Technical Indicator Assembly Document for NOAA NaMES}
\author{Willem Klajbor, The NOAA Ecosystem Indicators Working Group}
\date{2021-07-22}

\begin{document}
\maketitle

{
\setcounter{tocdepth}{1}
\tableofcontents
}
\hypertarget{overview}{%
\chapter*{Overview}\label{overview}}
\addcontentsline{toc}{chapter}{Overview}

The National Marine Ecosystem Status web portal provides the status of marine ecosystems across the U.S. and access to NOAA ecosystem indicator information and data. This website is designed to document the data sources and methods used to create the indicators displayed on the site.

\hypertarget{definition-of-indicators}{%
\section{Definition of Indicators}\label{definition-of-indicators}}

Ecosystem indicators are quantitative and/or qualitative measures of key components of the ecosystem. Marine ecosystems provide food, jobs, security, well-being, and other services to millions of people across the U.S. Yet, marine ecosystems and the people that rely on them are facing increasingly complex challenges. Tracking the status and trends of ocean and coastal ecosystems is critically important to understand how these ecosystems are changing and identify potential issues.

\hypertarget{chlorophyll-a}{%
\chapter{Chlorophyll-a}\label{chlorophyll-a}}

\hypertarget{data}{%
\section{Data}\label{data}}

Under Construction

\hypertarget{methods}{%
\section{Methods}\label{methods}}

Under construction

\hypertarget{zooplankton}{%
\chapter{Zooplankton}\label{zooplankton}}

\hypertarget{data-1}{%
\section{Data}\label{data-1}}

Under Construction

\hypertarget{methods-1}{%
\section{Methods}\label{methods-1}}

Under construction

\hypertarget{coral-reefs}{%
\chapter{Coral Reefs}\label{coral-reefs}}

\hypertarget{data-2}{%
\section{Data}\label{data-2}}

Under Construction

\hypertarget{methods-2}{%
\section{Methods}\label{methods-2}}

Under construction

\hypertarget{forage-fish}{%
\chapter{Forage Fish}\label{forage-fish}}

Forage fish, otherwise known as small pelagics are fish and invertebrates (like squids) that inhabit - the pelagic zone - the open ocean. The number and distribution of pelagic fish vary regionally, depending on multiple physical and ecological factors i.e.~the availability of light, nutrients, dissolved oxygen, temperature, salinity, predation, abundance of phytoplankton and zooplankton, etc. Small pelagics are known to exhibit ``boom and bust'' cycles of abundance in response to these conditions. Examples include anchovies, sardines, shad, menhaden and the fish that feed on them

Small pelagic species are often important to fisheries and serve as forage for commercially and recreationally important fish, as well as other ecosystem species (e.g.~seabirds and marine mammals). They are a critical part of marine food webs and important to monitor because so many other organisms depend on them. We present the annual total biomass of small pelagics/forage fish in the Alaska, California Current, and Northeast regions, as well as selected taxa in the Gulf of Mexico region.

\hypertarget{data-3}{%
\section{Data}\label{data-3}}

\hypertarget{alaska}{%
\subsection{Alaska}\label{alaska}}

The Indicator for the NaMES Alaksa regions is the East Bering Sea Pelagic forager biomass (fish 1000t) which can be directly accessed from the AKIEA website here: \url{https://apps-afsc.fisheries.noaa.gov/refm/reem/ecoweb/Index.php?ID=9}

\hypertarget{california-current}{%
\subsection{California Current}\label{california-current}}

\hypertarget{gulf-of-mexico}{%
\subsection{Gulf of Mexico}\label{gulf-of-mexico}}

The GoA IEA team produced a standardized menhaden biomass variable that is stored at \url{https://github.com/mandykarnauskas/GoM-Ecosystem-Status-Report/blob/master/data/menhaden_abundance_index.csv} As of this writing, the data had not been updated since 2020 and only go until 2015.

\hypertarget{northeast}{%
\subsection{Northeast}\label{northeast}}

The NE IEA Program keeps all of their ESR data in an R package called ecodata. The following code will download and install ecodata to your machine.

\begin{Shaded}
\begin{Highlighting}[]
\NormalTok{remotes}\SpecialCharTok{::}\FunctionTok{install\_github}\NormalTok{(}\StringTok{"NOAA{-}EDAB/ecodata"}\NormalTok{)}
\FunctionTok{library}\NormalTok{(ecodata)}
\FunctionTok{library}\NormalTok{(dplyr)}
\FunctionTok{library}\NormalTok{(tidyr)}
\FunctionTok{library}\NormalTok{(stringr)}
\DocumentationTok{\#\# These plots were pulled from: https://github.com/NOAA{-}EDAB/SOE{-}NEFMC/blob/master/SOE{-}NEFMC{-}2019.Rmd and https://github.com/NOAA{-}EDAB/SOE{-}MAFMC/blob/master/SOE{-}MAFMC{-}2019.Rmd}
\end{Highlighting}
\end{Shaded}

\hypertarget{southeast}{%
\subsection{Southeast}\label{southeast}}

The Southeast Atlantic forage indicator is the Menhaden CPUE Index score used by the SEIEA team in their Ecosystem Status Report. At the time of writing, the report is under review and data are not publicly available, but can be found on our Google Drive and GitHub Repository. This will be updated in the future once data are available.

\hypertarget{methods-3}{%
\section{Methods}\label{methods-3}}

\hypertarget{alaska-1}{%
\subsection{Alaska}\label{alaska-1}}

The variable can be directly downloaded from this link: \url{https://apps-afsc.fisheries.noaa.gov/refm/reem/ecoweb/csv/table/Pelagic.csv}

\hypertarget{california-current-1}{%
\subsection{California Current}\label{california-current-1}}

\hypertarget{gulf-of-mexico-1}{%
\subsection{Gulf of Mexico}\label{gulf-of-mexico-1}}

The GoM Menhaden Biomass (age 1+) data can be directly accessed and downloaded at this link: \url{https://github.com/mandykarnauskas/GoM-Ecosystem-Status-Report/blob/master/data/menhaden_abundance_index.csv}

\hypertarget{northeast-1}{%
\subsection{Northeast}\label{northeast-1}}

After downloading, installing, and loading the ecodata package (see above), use the following code to access the planktivore biomass data:

\begin{Shaded}
\begin{Highlighting}[]
\DocumentationTok{\#\# MA{-}SOE Fig. 16, NE{-}SOE Fig. 17{-}18}
\NormalTok{total\_surv }\OtherTok{\textless{}{-}}\NormalTok{ ecodata}\SpecialCharTok{::}\NormalTok{nefsc\_survey }\SpecialCharTok{\%\textgreater{}\%}
  \FunctionTok{filter}\NormalTok{(EPU }\SpecialCharTok{\%in\%} \FunctionTok{c}\NormalTok{(}\StringTok{"MAB"}\NormalTok{),}
         \SpecialCharTok{!}\FunctionTok{str\_detect}\NormalTok{(Var, }\StringTok{"Other|Apex|managed"}\NormalTok{),}
\NormalTok{         Time }\SpecialCharTok{\textgreater{}=} \DecValTok{1968}\NormalTok{) }\SpecialCharTok{\%\textgreater{}\%}
  \FunctionTok{mutate}\NormalTok{(}\AttributeTok{Var =} \FunctionTok{word}\NormalTok{(Var, }\DecValTok{1}\NormalTok{,}\DecValTok{2}\NormalTok{))}

\NormalTok{total\_surv2 }\OtherTok{\textless{}{-}} \FunctionTok{filter}\NormalTok{(total\_surv, Var }\SpecialCharTok{==} \StringTok{"Planktivore Spring"}\NormalTok{)}

\FunctionTok{write.csv}\NormalTok{(total\_surv2, }\AttributeTok{file =} \StringTok{"\textasciitilde{}/total\_surveyed\_biomass.csv"}\NormalTok{, }\AttributeTok{row.names =} \ConstantTok{FALSE}\NormalTok{)}
\end{Highlighting}
\end{Shaded}

This produces a .csv file with the name ``total\_surv2'' that displays The Spring Planktivore biomass in kg \^{}tow -1, which is the indicator used for NaMES.

\hypertarget{seabirds}{%
\chapter{Seabirds}\label{seabirds}}

Seabirds are a vital part of marine ecosystems and valuable indicators of an ecosystem's status. Seabirds are attracted to fishing vessels and frequently get hooked or entangled in fishing gear, especially longline fisheries. This is a common threat to seabirds. Depending on the geographic region, fishermen in the United States often interact with albatross, cormorants, gannet, loons, pelicans, puffins, gulls, storm-petrels, shearwaters, terns, and many other species. We track seabirds because of their importance to marine food webs, but also as an indication of efficient fishing practices. We present estimates of seabird abundance in the Alaska, California Current, Gulf of Mexico and Northeast regions.

\hypertarget{data-4}{%
\section{Data}\label{data-4}}

Data for Alaska, California Current, and the Gulf of Mexico were obtained from the regional NOAA Integrated Ecosystem Assessment Program teams that produce indicators and Ecosystem Status Report. The links for each of the datasets can be found here:

Alaska: \url{https://apps-afsc.fisheries.noaa.gov/refm/reem/ecoweb/csv/table/Seabird.csv}

California Current: \url{https://oceanview.pfeg.noaa.gov/erddap/tabledap/cciea_B_AS_DENS.csv?time,density_anomaly\&species_cohort=\%22Cassins\%20auklet\%20(So\%20CC,\%20Spring)\%22}

Gulf of Mexico: \url{https://github.com/mandykarnauskas/GoM-Ecosystem-Status-Report/blob/master/data/bird_standardized_abundancesFINAL.csv}

Seabird count and transect length data for the Northeast are extracted from the Atlantic Marine Assessment Program for Protected Species (AMAPPS) annual reports. Counts are summed and divided by the sum of the transect length in nautical miles. For more information see \url{https://www.nefsc.noaa.gov/psb/AMAPPS/}

\hypertarget{methods-4}{%
\section{Methods}\label{methods-4}}

\hypertarget{alaska-2}{%
\subsection{Alaska}\label{alaska-2}}

The Multivariate Breeding Index variable from the Eastern Bering Sea Ecosystem Status Report is currently used to represent seabirds for the NaMES Alaska Region. That data can be downloaded manually by clicking \url{https://apps-afsc.fisheries.noaa.gov/refm/reem/ecoweb/csv/table/Seabird.csv}.

\hypertarget{california-current-2}{%
\subsection{California Current}\label{california-current-2}}

The Density Anomaly for Cassin's Auklets in the Southern California Current (Spring) are used to represent the seabird indicator for the NaMES California Current region. Because this is an anomaly variable, the values for previous years will change every year - therefore, the entire dataset must be downloaded and replaced each year. The data can be downloaded manually by clicking \url{https://oceanview.pfeg.noaa.gov/erddap/tabledap/cciea_B_AS_DENS.csv?time,density_anomaly\&species_cohort=\%22Cassins\%20auklet\%20(So\%20CC,\%20Spring)\%22}

\hypertarget{gulf-of-mexico-2}{%
\subsection{Gulf of Mexico}\label{gulf-of-mexico-2}}

The GoA IEA team produces a standardized seabird relative abundance variable that is stored at \url{https://github.com/mandykarnauskas/GoM-Ecosystem-Status-Report/blob/master/data/bird_standardized_abundancesFINAL.csv}. As of this writing, the data had not been updated since 2020 and only go until 2015.

\hypertarget{northeast-2}{%
\subsection{Northeast}\label{northeast-2}}

The northeast seabirds indicator is compiled using the AMAPPS Annual report each year. This indicator is compiled completely manually. The reports can be found at \url{https://www.nefsc.noaa.gov/psb/AMAPPS/}. To calculate the indicator score for each year, one must go to the Seabirds -\textgreater{} Results section of a corresponding year's report and identify the paragraphs containing the ``Total seabirds seen,'' ``Total Seen in Zone,'' and the ``Nautical Miles Surveyed'' values. Then, those values for each cruise should be input into this google sheet (\url{https://docs.google.com/spreadsheets/d/1-g_d9eMWUchbm2xojlRG8Q_y7xzbtcyU1bHriW2ua4U/edit\#gid=0}). Finally, using the google sheet, the total number of seabirds observed for the entire year should be divided by the total nautical miles surveyed. That result produces the indicator score and should be manually input into the data file.

\hypertarget{overfished-stocks}{%
\chapter{Overfished Stocks}\label{overfished-stocks}}

Fish play an important role in marine ecosystems, supporting the ecological structure of many marine food webs. Caught by recreational and commercial fisheries, fish support significant parts of coastal economies, and can play an important cultural role in many regions. To understand the health of fish populations - as well as their abundance and distribution, we regularly assess fish stocks - stock assessments. Assessments let us know if a stock is experiencing overfishing or if it is overfished i.e.~how much catch is sustainable while maintaining a healthy stock. And, if a stock becomes depleted, stock assessments can help determine what steps may be taken to rebuild it to sustainable levels. Understanding stock assessments helps measure how well we're managing and recovering fish stocks over time.

We present the number of overfished stocks by year in all regions.

\hypertarget{data-5}{%
\section{Data}\label{data-5}}

Data are obtained from the NOAA Fisheries Fishery Stock Status website \url{https://www.fisheries.noaa.gov/national/population-assessments/fishery-stock-status-updates}. Stocks that meet the criteria for overfished status are summed by year for each region. The status of stocks are available in report form and graphically.

\hypertarget{methods-5}{%
\section{Methods}\label{methods-5}}

This indicator is compiled manually.

After traveling to \url{https://www.fisheries.noaa.gov/national/population-assessments/fishery-stock-status-updates}, identify the most up to date status report. The reports should be available in two formats - through a report and through a visual.

The easiest way to compile the data are using the visual, which should be named ``Stocks on the Overfished and Overfishing Lists by Region.'' After opening the visual (example below), overfished stocks will be displayed by sqaures and sorted spatially by region. We are only counting the ``Overfished'' stocks - do not count stocks that are only on the ``Overfishing'' list.

\begin{Shaded}
\begin{Highlighting}[]
\NormalTok{knitr}\SpecialCharTok{::}\FunctionTok{include\_graphics}\NormalTok{(}\FunctionTok{rep}\NormalTok{(}\StringTok{"overfished.png"}\NormalTok{))}
\end{Highlighting}
\end{Shaded}

\begin{figure}
\includegraphics[width=38.39in]{overfished} \caption{NOAA Fisheries 2020 Q4 Stock Status Map}\label{fig:unnamed-chunk-4}
\end{figure}

North Pacific corresponds to the Alaska Region, Pacific corresponds to the California Current, Western Pacific corresponds to Hawaii (but be sure not to count the pacific island specific complexes), and New England corresponds to North Atlantic.

For more information, contact Willem Klajbor (\href{mailto:willem.klajbor@noaa.gov}{\nolinkurl{willem.klajbor@noaa.gov}}) or Stephanie Oakes (\href{mailto:stephanie.oakes@noaa.gov}{\nolinkurl{stephanie.oakes@noaa.gov}}).

\hypertarget{marine-mammals}{%
\chapter{Marine Mammals}\label{marine-mammals}}

\hypertarget{esa}{%
\section{ESA}\label{esa}}

\hypertarget{data-6}{%
\subsection{Data}\label{data-6}}

Under Construction

\hypertarget{methods-6}{%
\subsection{Methods}\label{methods-6}}

Under construction

\hypertarget{mmpa}{%
\section{MMPA}\label{mmpa}}

\hypertarget{data-7}{%
\subsection{Data}\label{data-7}}

Under Construction

\hypertarget{methods-7}{%
\subsection{Methods}\label{methods-7}}

Under construction

\hypertarget{unusual-mortality-events}{%
\chapter{Unusual Mortality Events}\label{unusual-mortality-events}}

\hypertarget{data-8}{%
\section{Data}\label{data-8}}

Under Construction

\hypertarget{methods-8}{%
\section{Methods}\label{methods-8}}

Under construction

\hypertarget{sea-surface-temperature}{%
\chapter{Sea Surface Temperature}\label{sea-surface-temperature}}

\hypertarget{data-9}{%
\section{Data}\label{data-9}}

Under Construction

\hypertarget{methods-9}{%
\section{Methods}\label{methods-9}}

Under construction

\hypertarget{sea-level}{%
\chapter{Sea Level}\label{sea-level}}

\hypertarget{data-10}{%
\section{Data}\label{data-10}}

Under Construction

\hypertarget{methods-10}{%
\section{Methods}\label{methods-10}}

Under construction

\hypertarget{sea-ice}{%
\chapter{Sea Ice}\label{sea-ice}}

Unlike icebergs, glaciers, ice sheets, and ice shelves, which originate on land, sea ice forms, expands, and melts in the ocean. Sea ice influences global climate by reflecting sunlight back into space. Because this solar energy is not absorbed into the ocean, temperatures nearer the poles remain cool. When sea ice melts, the surface area reflecting sunlight decreases, allowing more solar energy to be absorbed by the ocean, causing temperatures to rise. This creates a positive feedback loop. Warmer water temperatures delay ice growth in the autumn and winter, and the ice melts faster the following spring, exposing dark ocean waters for longer periods the following summer.

Sea ice affects the movement of ocean waters. When sea ice forms, ocean salts are left behind. As the seawater gets saltier, its density increases, and it sinks. Surface water is pulled in to replace the sinking water, which in turn becomes cold and salty and sinks. This initiates deep-ocean currents driving the global ocean conveyor belt.

Sea ice is an important element of the Arctic system. It provides an important habitat for biological activity, i.e.~algae grows on the bottom of sea ice, forming the basis of the Arctic food web, and it plays a critical role in the life cycle of many marine mammals - seals and polar bears. Sea ice also serves a critical role in supporting Indigenous communities culture and survival. We present the annual sea ice extent in millions of Kilometers for the Arctic region.

\hypertarget{data-11}{%
\section{Data}\label{data-11}}

Sea ice data was accessed from the NOAA National Centers for Environmental Information, \url{https://www.ncdc.noaa.gov/snow-and-ice/extent/} , with the data pulled from here: \url{https://www.ncdc.noaa.gov/snow-and-ice/extent/sea-ice/N/3/data.csv}. The data are plotted in units of million square km.

\hypertarget{methods-11}{%
\section{Methods}\label{methods-11}}

To download the current sea ice data, you can either:

\begin{enumerate}
\def\labelenumi{\arabic{enumi})}
\tightlist
\item
  Copy/paste the following url into your web browser:
  \url{https://www.ncdc.noaa.gov/snow-and-ice/extent/sea-ice/N/3/data.csv}
\end{enumerate}

or

\begin{enumerate}
\def\labelenumi{\arabic{enumi})}
\setcounter{enumi}{1}
\tightlist
\item
  Use the following R code to download the data and import it into your RStudio environment
\end{enumerate}

\begin{Shaded}
\begin{Highlighting}[]
\NormalTok{url }\OtherTok{\textless{}{-}}\StringTok{"https://www.ncdc.noaa.gov/snow{-}and{-}ice/extent/sea{-}ice/N/3/data.csv"}
\CommentTok{\# Specify destination where file should be saved}
\NormalTok{destfile }\OtherTok{\textless{}{-}} \StringTok{"C:/Users/ ... Your Path ... /my folder/output.csv"}
\CommentTok{\#Apply download.file function in R}
\FunctionTok{download.file}\NormalTok{(url, destfile)}
\end{Highlighting}
\end{Shaded}

Data were restructured and gauge values were calculated manually.

For more information, contact Willem Klajbor (\href{mailto:willem.klajbor@noaa.gov}{\nolinkurl{willem.klajbor@noaa.gov}}) or Scott Cross (\href{mailto:scott.cross@noaa.gov}{\nolinkurl{scott.cross@noaa.gov}}).

\hypertarget{climate-indices}{%
\chapter{Climate Indices}\label{climate-indices}}

\hypertarget{enso}{%
\section{ENSO}\label{enso}}

Under Construction

\hypertarget{mei}{%
\section{MEI}\label{mei}}

Under construction

\hypertarget{pdo}{%
\section{PDO}\label{pdo}}

Under Construction

\hypertarget{epnp}{%
\section{EPNP}\label{epnp}}

Under Construction

\hypertarget{nao}{%
\section{NAO}\label{nao}}

Under Construction

\hypertarget{amo}{%
\section{AMO}\label{amo}}

Under Construction

\hypertarget{coastal-population}{%
\chapter{Coastal Population}\label{coastal-population}}

\hypertarget{data-12}{%
\section{Data}\label{data-12}}

Under Construction

\hypertarget{methods-12}{%
\section{Methods}\label{methods-12}}

Under construction

\hypertarget{coastal-tourism}{%
\chapter{Coastal Tourism}\label{coastal-tourism}}

\hypertarget{data-13}{%
\section{Data}\label{data-13}}

Under Construction

\hypertarget{methods-13}{%
\section{Methods}\label{methods-13}}

Under construction

\hypertarget{coastal-employment}{%
\chapter{Coastal Employment}\label{coastal-employment}}

\hypertarget{data-14}{%
\section{Data}\label{data-14}}

Under Construction

\hypertarget{methods-14}{%
\section{Methods}\label{methods-14}}

Under construction

\hypertarget{commercial-fishing}{%
\chapter{Commercial Fishing}\label{commercial-fishing}}

\hypertarget{landings}{%
\section{Landings}\label{landings}}

\hypertarget{data-15}{%
\subsection{Data}\label{data-15}}

Under Construction

\hypertarget{methods-15}{%
\subsection{Methods}\label{methods-15}}

Under construction

\hypertarget{revenue}{%
\section{Revenue}\label{revenue}}

\hypertarget{data-16}{%
\subsection{Data}\label{data-16}}

Under Construction

\hypertarget{methods-16}{%
\subsection{Methods}\label{methods-16}}

Under construction

\hypertarget{recreational-fishing}{%
\chapter{Recreational Fishing}\label{recreational-fishing}}

\hypertarget{effort}{%
\section{Effort}\label{effort}}

\hypertarget{data-17}{%
\subsection{Data}\label{data-17}}

Under Construction

\hypertarget{methods-17}{%
\subsection{Methods}\label{methods-17}}

Under construction

\hypertarget{harvest}{%
\section{Harvest}\label{harvest}}

\hypertarget{data-18}{%
\subsection{Data}\label{data-18}}

Under Construction

\hypertarget{methods-18}{%
\subsection{Methods}\label{methods-18}}

Under construction

\hypertarget{fishing-engagement}{%
\chapter{Fishing Engagement}\label{fishing-engagement}}

\hypertarget{commercial}{%
\section{Commercial}\label{commercial}}

\hypertarget{data-19}{%
\subsection{Data}\label{data-19}}

Under Construction

\hypertarget{methods-19}{%
\subsection{Methods}\label{methods-19}}

Under construction

\hypertarget{recreational}{%
\section{Recreational}\label{recreational}}

\hypertarget{data-20}{%
\subsection{Data}\label{data-20}}

Under Construction

\hypertarget{methods-20}{%
\subsection{Methods}\label{methods-20}}

Under construction

\hypertarget{billion-dollar-disasters}{%
\chapter{Billion Dollar Disasters}\label{billion-dollar-disasters}}

In the United States the number of weather and climate-related disasters exceeding 1 billion dollars has been increasing since 1980. These events have significant impacts to coastal economies and communities. The Billion Dollar Disaster indicator provides information on the frequency and the total estimated costs of major weather and climate events that occur in the United States. This indicator compiles the annual number of weather and climate-related disasters across seven event types. Events are included if they are estimated to cause more than one billion U.S. dollars in direct losses. The cost estimates of these events are adjusted for inflation using the Consumer Price Index (CPI) and are based on costs documented in several Federal and private-sector databases. We present the total annual number of disaster events for all regions.

\hypertarget{data-21}{%
\section{Data}\label{data-21}}

Billion dollar disaster event frequency data are taken from NOAA's National Centers for Environmental Information (\url{https://www.ncdc.noaa.gov/billions/}). The number of disasters within each region were summed for every year of available data. Although the number is the count of unique disaster events within a region, the same disaster can impact multiple regions, meaning a sum across regions will overestimate the unique number of disasters.

\hypertarget{methods-21}{%
\section{Methods}\label{methods-21}}

The Billion Dollar Event Frequency Data displayed on the website were compiled using the following code:

\begin{Shaded}
\begin{Highlighting}[]
\NormalTok{PKG }\OtherTok{\textless{}{-}} \FunctionTok{c}\NormalTok{(}\StringTok{"foreign"}\NormalTok{,}\StringTok{"stringr"}\NormalTok{,}\StringTok{"data.table"}\NormalTok{)}

\ControlFlowTok{for}\NormalTok{ (p }\ControlFlowTok{in}\NormalTok{ PKG) \{}
  \ControlFlowTok{if}\NormalTok{(}\SpecialCharTok{!}\FunctionTok{require}\NormalTok{(p,}\AttributeTok{character.only =} \ConstantTok{TRUE}\NormalTok{)) \{  }
    \FunctionTok{install.packages}\NormalTok{(p)}
    \FunctionTok{require}\NormalTok{(p,}\AttributeTok{character.only =} \ConstantTok{TRUE}\NormalTok{)\}}
\NormalTok{\}}

\CommentTok{\#states \textless{}{-} c("AK","AL","AR","AZ","CA","CO","CT","DE","FL","GA","HI",}
\CommentTok{\#            "IA","ID","IL","IN","KS","KY","LA","MA","MD","ME","MI",}
\CommentTok{\#            "MN","MO","MS","MT","NC","ND","NE","NH","NJ","NM","NV",}
\CommentTok{\#            "NY","OH","OK","OR","PA","RI","SC","SD","TN","TX","UT",}
\CommentTok{\#           "VA","VT","WA","WI","WV","WY")}

\NormalTok{states }\OtherTok{\textless{}{-}} \FunctionTok{c}\NormalTok{(}\StringTok{"AK"}\NormalTok{,}\StringTok{"AL"}\NormalTok{,}\StringTok{"CA"}\NormalTok{,}\StringTok{"CT"}\NormalTok{,}\StringTok{"DE"}\NormalTok{,}\StringTok{"FL"}\NormalTok{,}\StringTok{"GA"}\NormalTok{,}\StringTok{"HI"}\NormalTok{,}
            \StringTok{"LA"}\NormalTok{,}\StringTok{"MA"}\NormalTok{,}\StringTok{"MD"}\NormalTok{,}\StringTok{"ME"}\NormalTok{,}
            \StringTok{"MS"}\NormalTok{,}\StringTok{"NC"}\NormalTok{,}\StringTok{"NH"}\NormalTok{,}\StringTok{"NJ"}\NormalTok{,}
            \StringTok{"NY"}\NormalTok{,}\StringTok{"OR"}\NormalTok{,}\StringTok{"PA"}\NormalTok{,}\StringTok{"RI"}\NormalTok{,}\StringTok{"SC"}\NormalTok{,}\StringTok{"TX"}\NormalTok{,}
            \StringTok{"VA"}\NormalTok{,}\StringTok{"WA"}\NormalTok{,}\StringTok{"PR"}\NormalTok{,}\StringTok{"VI"}\NormalTok{)}

\CommentTok{\#Update Year in URL (2021)}
\NormalTok{Billion\_Storm }\OtherTok{\textless{}{-}} \ConstantTok{NULL}
\ControlFlowTok{for}\NormalTok{ (x }\ControlFlowTok{in}\NormalTok{ states) \{}
\NormalTok{  temp }\OtherTok{\textless{}{-}} \FunctionTok{tempfile}\NormalTok{()}
\NormalTok{  temp.connect }\OtherTok{\textless{}{-}} \FunctionTok{url}\NormalTok{(}\FunctionTok{paste0}\NormalTok{(}\StringTok{"https://www.ncdc.noaa.gov/billions/events{-}"}\NormalTok{,x,}\StringTok{"{-}1980{-}2021.csv"}\NormalTok{, }\AttributeTok{sep=}\StringTok{""}\NormalTok{))}
\NormalTok{  temp }\OtherTok{\textless{}{-}} \FunctionTok{data.table}\NormalTok{(}\FunctionTok{read.delim}\NormalTok{(temp.connect, }\AttributeTok{header=}\ConstantTok{TRUE}\NormalTok{,}\AttributeTok{fill=}\ConstantTok{FALSE}\NormalTok{, }\AttributeTok{stringsAsFactors=}\ConstantTok{FALSE}\NormalTok{,}\AttributeTok{skip=}\DecValTok{1}\NormalTok{, }\AttributeTok{sep=}\StringTok{","}\NormalTok{))}
\NormalTok{  temp}\SpecialCharTok{$}\NormalTok{State }\OtherTok{\textless{}{-}}\NormalTok{ x}
\NormalTok{  Billion\_Storm }\OtherTok{\textless{}{-}} \FunctionTok{rbind}\NormalTok{(Billion\_Storm,temp)}
  \FunctionTok{unlink}\NormalTok{(temp)}
  \FunctionTok{rm}\NormalTok{(temp)}
\NormalTok{\}}

\NormalTok{Billion\_Storm}\SpecialCharTok{$}\NormalTok{Begin.Date }\OtherTok{\textless{}{-}} \FunctionTok{as.character}\NormalTok{(Billion\_Storm}\SpecialCharTok{$}\NormalTok{Begin.Date)}
\NormalTok{Billion\_Storm}\SpecialCharTok{$}\NormalTok{Begin.Year }\OtherTok{\textless{}{-}}  \FunctionTok{substr}\NormalTok{(Billion\_Storm}\SpecialCharTok{$}\NormalTok{Begin.Date,}\DecValTok{1}\NormalTok{,}\DecValTok{4}\NormalTok{)}
\NormalTok{Billion\_Storm}\SpecialCharTok{$}\NormalTok{Begin.Date }\OtherTok{\textless{}{-}} \FunctionTok{as.Date}\NormalTok{(Billion\_Storm}\SpecialCharTok{$}\NormalTok{Begin.Date,}\StringTok{"\%Y \%m \%d"}\NormalTok{)}
\NormalTok{Billion\_Storm}\SpecialCharTok{$}\NormalTok{End.Date }\OtherTok{\textless{}{-}} \FunctionTok{as.character}\NormalTok{(Billion\_Storm}\SpecialCharTok{$}\NormalTok{End.Date)}
\NormalTok{Billion\_Storm}\SpecialCharTok{$}\NormalTok{End.Year }\OtherTok{\textless{}{-}}  \FunctionTok{substr}\NormalTok{(Billion\_Storm}\SpecialCharTok{$}\NormalTok{End.Date,}\DecValTok{1}\NormalTok{,}\DecValTok{4}\NormalTok{)}
\NormalTok{Billion\_Storm}\SpecialCharTok{$}\NormalTok{End.Date }\OtherTok{\textless{}{-}} \FunctionTok{as.Date}\NormalTok{(Billion\_Storm}\SpecialCharTok{$}\NormalTok{End.Date,}\StringTok{"\%Y \%m \%d"}\NormalTok{)}

\NormalTok{Gulf.of.Mexico }\OtherTok{\textless{}{-}} \FunctionTok{c}\NormalTok{(}\StringTok{"FL"}\NormalTok{,}\StringTok{"AL"}\NormalTok{,}\StringTok{"LA"}\NormalTok{,}\StringTok{"MS"}\NormalTok{,}\StringTok{"TX"}\NormalTok{)}
\NormalTok{Northeast }\OtherTok{\textless{}{-}} \FunctionTok{c}\NormalTok{(}\StringTok{"NC"}\NormalTok{,}\StringTok{"VA"}\NormalTok{,}\StringTok{"MD"}\NormalTok{,}\StringTok{"DE"}\NormalTok{,}\StringTok{"PA"}\NormalTok{,}\StringTok{"NJ"}\NormalTok{,}\StringTok{"NY"}\NormalTok{,}\StringTok{"CT"}\NormalTok{,}\StringTok{"RI"}\NormalTok{,}
               \StringTok{"MA"}\NormalTok{,}\StringTok{"NH"}\NormalTok{,}\StringTok{"ME"}\NormalTok{)}
\NormalTok{Southeast }\OtherTok{\textless{}{-}} \FunctionTok{c}\NormalTok{(}\StringTok{"SC"}\NormalTok{,}\StringTok{"GA"}\NormalTok{,}\StringTok{"FL"}\NormalTok{)}
\NormalTok{California.Current }\OtherTok{\textless{}{-}} \FunctionTok{c}\NormalTok{(}\StringTok{"CA"}\NormalTok{,}\StringTok{"OR"}\NormalTok{,}\StringTok{"WA"}\NormalTok{)}
\NormalTok{Alaska}\OtherTok{\textless{}{-}} \FunctionTok{c}\NormalTok{(}\StringTok{"AK"}\NormalTok{)}
\NormalTok{Hawaii }\OtherTok{\textless{}{-}} \FunctionTok{c}\NormalTok{(}\StringTok{"HI"}\NormalTok{)}
\NormalTok{Caribbean }\OtherTok{\textless{}{-}} \FunctionTok{c}\NormalTok{(}\StringTok{"PR"}\NormalTok{,}\StringTok{"VI"}\NormalTok{)}

\NormalTok{Storm\_Freq }\OtherTok{\textless{}{-}} \ConstantTok{NULL}
\ControlFlowTok{for}\NormalTok{ (x }\ControlFlowTok{in} \FunctionTok{c}\NormalTok{(}\StringTok{"Gulf.of.Mexico"}\NormalTok{,}\StringTok{"Northeast"}\NormalTok{,}\StringTok{"Southeast"}\NormalTok{,}\StringTok{"California.Current"}\NormalTok{,}\StringTok{"Alaska"}\NormalTok{,}\StringTok{"Hawaii"}\NormalTok{,}\StringTok{"Caribbean"}\NormalTok{)) \{}
\NormalTok{  TEMP }\OtherTok{\textless{}{-}}\NormalTok{ Billion\_Storm[}\FunctionTok{which}\NormalTok{(Billion\_Storm}\SpecialCharTok{$}\NormalTok{State}\SpecialCharTok{\%in\%}\FunctionTok{get}\NormalTok{(x)),]}
\NormalTok{  TEMP}\SpecialCharTok{$}\NormalTok{Disaster }\OtherTok{\textless{}{-}}\NormalTok{ TEMP}\SpecialCharTok{$}\NormalTok{Begin.Date }\OtherTok{\textless{}{-}}\NormalTok{ TEMP}\SpecialCharTok{$}\NormalTok{End.Date }\OtherTok{\textless{}{-}}\NormalTok{ TEMP}\SpecialCharTok{$}\NormalTok{Deaths }\OtherTok{\textless{}{-}}\NormalTok{ TEMP}\SpecialCharTok{$}\NormalTok{State }\OtherTok{\textless{}{-}}\NormalTok{ TEMP}\SpecialCharTok{$}\NormalTok{Begin.Year }\OtherTok{\textless{}{-}} \ConstantTok{NULL}
\NormalTok{  TEMP }\OtherTok{\textless{}{-}} \FunctionTok{unique}\NormalTok{(TEMP)}
  \FunctionTok{colnames}\NormalTok{(TEMP)}\OtherTok{\textless{}{-}} \FunctionTok{c}\NormalTok{(}\StringTok{\textquotesingle{}Name\textquotesingle{}}\NormalTok{,}\StringTok{\textquotesingle{}Frequency\textquotesingle{}}\NormalTok{,}\StringTok{\textquotesingle{}End.Year\textquotesingle{}}\NormalTok{)}
\NormalTok{  TEMP }\OtherTok{\textless{}{-}} \FunctionTok{aggregate}\NormalTok{(Frequency}\SpecialCharTok{\textasciitilde{}}\NormalTok{End.Year, }\AttributeTok{data=}\NormalTok{TEMP, }\AttributeTok{FUN=}\NormalTok{length)}
\NormalTok{  TEMP}\SpecialCharTok{$}\NormalTok{Region }\OtherTok{\textless{}{-}}\NormalTok{ x}
  \FunctionTok{assign}\NormalTok{(}\FunctionTok{paste0}\NormalTok{(x,}\StringTok{"\_Data"}\NormalTok{, }\AttributeTok{sep=}\StringTok{""}\NormalTok{),TEMP)}
\NormalTok{  Storm\_Freq }\OtherTok{\textless{}{-}} \FunctionTok{rbind}\NormalTok{(Storm\_Freq,TEMP)}
  \FunctionTok{rm}\NormalTok{(TEMP)}
\NormalTok{\}}

\NormalTok{Storm\_Freq\_F }\OtherTok{\textless{}{-}} \FunctionTok{spread}\NormalTok{(Storm\_Freq,Region,Frequency)}

\FunctionTok{write.csv}\NormalTok{(Storm\_Freq\_F,}\AttributeTok{file=}\StringTok{"C:/Users/... your path.../Billion\_Dollar\_Storms\_1980\_Present.csv"}\NormalTok{)}
\FunctionTok{rm}\NormalTok{(}\AttributeTok{list=}\FunctionTok{ls}\NormalTok{())}
\end{Highlighting}
\end{Shaded}

Gauge values counted manually.

For more information, please contact Willem Klajbor (\href{mailto:willem.klajbor@noaa.gov}{\nolinkurl{willem.klajbor@noaa.gov}}) or Kate Quigley (\href{mailto:kate.quigley@noaa.gov}{\nolinkurl{kate.quigley@noaa.gov}}).

\hypertarget{beach-closures}{%
\chapter{Beach Closures}\label{beach-closures}}

Beach closures are the number of days when beach water and/or air quality is determined to be unsafe. Unsafe water and air quality may have significant impacts on human health, local economies, and the ecosystem. The Environmental Protection Agency (EPA) supports coastal states, counties and and tribes in monitoring beach water quality, and notifying the public when beaches must be closed. Beach water quality is determined by the concentration of bacteria in the water (either Enterococcus sp.or Escherichia coli).

The information presented is from states, counties, and tribes that submit data to the EPA Beach Program reporting database. Not all US beach closures are captured in this database. We present a summary of known EPA Beach Program closure days by year for Alaska, California Current, Gulf of Mexico, Northeast, Hawai'ian Islands, and the Southeast regions.

\hypertarget{data-22}{%
\section{Data}\label{data-22}}

Data obtained from the EPA BEACON website have been provided to EPA by the coastal and Great Lakes states, tribes and territories that receive grants under the BEACH Act. Data was refined to closure, by state or territory, by year. Data compiled by states or territory and combined in regions defined as IEA regions except PI includes Guam and Marianas. Caribbean and South Atlantic data stand alone. Not all beaches in a state or territory are monitored through the EPA BEACH Act. Data for beaches monitored by state and local municipalities is not included. Changes in the number of beach closure days may be driven by changes in the number of beaches monitored under the BEACH Act versus by state and local municipalities.

Step by step instructions for obtaining these data from the EPA BEACON 2.0 Portal:

Go to BEACON 2.0 website reports page
\url{https://watersgeo.epa.gov/beacon2/reports.html}
Step 1 location filter - select National all States/Tribes/Territories
Step 2 Additional Filtering - select Calendar Year Grouping option and select all year or the years you need to add to update the indicator
Step 3 Report Selection - Choose Beach Actions (advisories and closures)
Actions drop down - choose filter
Column drop down - select Action Type
Operator drop down - select =
Expression Drop dpown - select Closure and Apply (note there is a permanent closure option which was not used in first indicator - explore)
Action drop down - choose download and select CSV

\hypertarget{methods-22}{%
\section{Methods}\label{methods-22}}

Data were compiled using the following code:

\begin{Shaded}
\begin{Highlighting}[]
\CommentTok{\#import data}
\NormalTok{beach }\OtherTok{\textless{}{-}} \FunctionTok{read\_excel}\NormalTok{(}\StringTok{"C:/Users/your\_path.xlsx"}\NormalTok{)}

\CommentTok{\#remove unnecessary data}
\NormalTok{drop }\OtherTok{\textless{}{-}} \FunctionTok{c}\NormalTok{(}\StringTok{"County"}\NormalTok{, }\StringTok{"Beach Id"}\NormalTok{, }\StringTok{"Beach Name"}\NormalTok{, }\StringTok{"Beach Status"}\NormalTok{, }\StringTok{"Swim Status"}\NormalTok{, }\StringTok{"Reported"}\NormalTok{, }\StringTok{"Station ID"}\NormalTok{,}
          \StringTok{"Activity ID"}\NormalTok{, }\StringTok{"Action Reasons"}\NormalTok{, }\StringTok{"Action Indicator"}\NormalTok{, }\StringTok{"Action Possible Source"}\NormalTok{)}
\NormalTok{beach }\OtherTok{=}\NormalTok{ beach\_actions[,}\SpecialCharTok{!}\NormalTok{(}\FunctionTok{names}\NormalTok{(beach\_actions) }\SpecialCharTok{\%in\%}\NormalTok{ drop)]}

\CommentTok{\#look for best way to subset}
\NormalTok{water }\OtherTok{\textless{}{-}} \FunctionTok{unique}\NormalTok{(beach[}\FunctionTok{c}\NormalTok{(}\StringTok{"Waterbody Name"}\NormalTok{)])}
\NormalTok{keep }\OtherTok{\textless{}{-}} \FunctionTok{c}\NormalTok{(}\StringTok{"Atlantic Ocean"}\NormalTok{, }\StringTok{"Pacific Ocean"}\NormalTok{, }\StringTok{"Gulf of Mexico"}\NormalTok{, }\StringTok{"Long Island Sound"}\NormalTok{, }\StringTok{"{-}"}\NormalTok{, }\StringTok{"Chesapeake Bay"}\NormalTok{)}
\NormalTok{beach2 }\OtherTok{\textless{}{-}}\NormalTok{ beach[beach}\SpecialCharTok{$}\StringTok{\textasciigrave{}}\AttributeTok{Waterbody Name}\StringTok{\textasciigrave{}} \SpecialCharTok{\%in\%}\NormalTok{ keep, ]}

\CommentTok{\#one more subset...}
\NormalTok{states }\OtherTok{\textless{}{-}} \FunctionTok{unique}\NormalTok{(beach2[}\FunctionTok{c}\NormalTok{(}\StringTok{"State"}\NormalTok{)])}
\NormalTok{beach3 }\OtherTok{\textless{}{-}}\NormalTok{ beach2[beach2}\SpecialCharTok{$}\NormalTok{State }\SpecialCharTok{!=}\StringTok{"IL"} \SpecialCharTok{\&}\NormalTok{ beach2}\SpecialCharTok{$}\NormalTok{State }\SpecialCharTok{!=}\StringTok{"IN"}\NormalTok{, ]}

\CommentTok{\#add region}
\NormalTok{beach3}\SpecialCharTok{$}\NormalTok{Region }\OtherTok{\textless{}{-}}\StringTok{""}

\CommentTok{\#and assign}
\NormalTok{beach3}\SpecialCharTok{$}\NormalTok{Region[beach3}\SpecialCharTok{$}\NormalTok{State}\SpecialCharTok{==}\StringTok{"AK"}\NormalTok{] }\OtherTok{\textless{}{-}} \StringTok{"Alaska"}
\NormalTok{beach3}\SpecialCharTok{$}\NormalTok{Region[beach3}\SpecialCharTok{$}\NormalTok{State}\SpecialCharTok{==}\StringTok{"HI"}\NormalTok{] }\OtherTok{\textless{}{-}} \StringTok{"Hawaii{-}Pacific Islands"}
\NormalTok{beach3}\SpecialCharTok{$}\NormalTok{Region[beach3}\SpecialCharTok{$}\NormalTok{State}\SpecialCharTok{==}\StringTok{"CA"}\NormalTok{] }\OtherTok{\textless{}{-}} \StringTok{"California Current"}
\NormalTok{beach3}\SpecialCharTok{$}\NormalTok{Region[beach3}\SpecialCharTok{$}\NormalTok{State}\SpecialCharTok{==}\StringTok{"OR"}\NormalTok{] }\OtherTok{\textless{}{-}} \StringTok{"California Current"}
\NormalTok{beach3}\SpecialCharTok{$}\NormalTok{Region[beach3}\SpecialCharTok{$}\NormalTok{State}\SpecialCharTok{==}\StringTok{"WA"}\NormalTok{] }\OtherTok{\textless{}{-}} \StringTok{"California Current"}
\NormalTok{beach3}\SpecialCharTok{$}\NormalTok{Region[beach3}\SpecialCharTok{$}\NormalTok{State}\SpecialCharTok{==}\StringTok{"TX"}\NormalTok{] }\OtherTok{\textless{}{-}} \StringTok{"Gulf of Mexico"}
\NormalTok{beach3}\SpecialCharTok{$}\NormalTok{Region[beach3}\SpecialCharTok{$}\NormalTok{State}\SpecialCharTok{==}\StringTok{"AL"}\NormalTok{] }\OtherTok{\textless{}{-}} \StringTok{"Gulf of Mexico"}
\NormalTok{beach3}\SpecialCharTok{$}\NormalTok{Region[beach3}\SpecialCharTok{$}\NormalTok{State}\SpecialCharTok{==}\StringTok{"LA"}\NormalTok{] }\OtherTok{\textless{}{-}} \StringTok{"Gulf of Mexico"}
\NormalTok{beach3}\SpecialCharTok{$}\NormalTok{Region[beach3}\SpecialCharTok{$}\NormalTok{State}\SpecialCharTok{==}\StringTok{"MS"}\NormalTok{] }\OtherTok{\textless{}{-}} \StringTok{"Gulf of Mexico"}
\NormalTok{beach3}\SpecialCharTok{$}\NormalTok{Region[beach3}\SpecialCharTok{$}\NormalTok{State}\SpecialCharTok{==}\StringTok{"FL"} \SpecialCharTok{\&}\NormalTok{ beach3}\SpecialCharTok{$}\StringTok{\textasciigrave{}}\AttributeTok{Waterbody Name}\StringTok{\textasciigrave{}}\SpecialCharTok{==}\StringTok{"Gulf of Mexico"}\NormalTok{] }\OtherTok{\textless{}{-}} \StringTok{"Gulf of Mexico"}
\NormalTok{beach3}\SpecialCharTok{$}\NormalTok{Region[beach3}\SpecialCharTok{$}\NormalTok{State}\SpecialCharTok{==}\StringTok{"FL"} \SpecialCharTok{\&}\NormalTok{ beach3}\SpecialCharTok{$}\StringTok{\textasciigrave{}}\AttributeTok{Waterbody Name}\StringTok{\textasciigrave{}}\SpecialCharTok{==}\StringTok{"Atlantic Ocean"}\NormalTok{] }\OtherTok{\textless{}{-}} \StringTok{"Southeast"}
\NormalTok{beach3}\SpecialCharTok{$}\NormalTok{Region[beach3}\SpecialCharTok{$}\NormalTok{State}\SpecialCharTok{==}\StringTok{"SC"}\NormalTok{] }\OtherTok{\textless{}{-}} \StringTok{"Southeast"}
\NormalTok{beach3}\SpecialCharTok{$}\NormalTok{Region[beach3}\SpecialCharTok{$}\NormalTok{State}\SpecialCharTok{==}\StringTok{"GA"}\NormalTok{] }\OtherTok{\textless{}{-}} \StringTok{"Southeast"}
\NormalTok{beach3}\SpecialCharTok{$}\NormalTok{Region[beach3}\SpecialCharTok{$}\NormalTok{State}\SpecialCharTok{==}\StringTok{"NC"}\NormalTok{] }\OtherTok{\textless{}{-}} \StringTok{"Northeast"}
\NormalTok{beach3}\SpecialCharTok{$}\NormalTok{Region[beach3}\SpecialCharTok{$}\NormalTok{State}\SpecialCharTok{==}\StringTok{"VA"}\NormalTok{] }\OtherTok{\textless{}{-}} \StringTok{"Northeast"}
\NormalTok{beach3}\SpecialCharTok{$}\NormalTok{Region[beach3}\SpecialCharTok{$}\NormalTok{State}\SpecialCharTok{==}\StringTok{"MD"}\NormalTok{] }\OtherTok{\textless{}{-}} \StringTok{"Northeast"}
\NormalTok{beach3}\SpecialCharTok{$}\NormalTok{Region[beach3}\SpecialCharTok{$}\NormalTok{State}\SpecialCharTok{==}\StringTok{"DE"}\NormalTok{] }\OtherTok{\textless{}{-}} \StringTok{"Northeast"}
\NormalTok{beach3}\SpecialCharTok{$}\NormalTok{Region[beach3}\SpecialCharTok{$}\NormalTok{State}\SpecialCharTok{==}\StringTok{"NJ"}\NormalTok{] }\OtherTok{\textless{}{-}} \StringTok{"Northeast"}
\NormalTok{beach3}\SpecialCharTok{$}\NormalTok{Region[beach3}\SpecialCharTok{$}\NormalTok{State}\SpecialCharTok{==}\StringTok{"NY"}\NormalTok{] }\OtherTok{\textless{}{-}} \StringTok{"Northeast"}
\NormalTok{beach3}\SpecialCharTok{$}\NormalTok{Region[beach3}\SpecialCharTok{$}\NormalTok{State}\SpecialCharTok{==}\StringTok{"CT"}\NormalTok{] }\OtherTok{\textless{}{-}} \StringTok{"Northeast"}
\NormalTok{beach3}\SpecialCharTok{$}\NormalTok{Region[beach3}\SpecialCharTok{$}\NormalTok{State}\SpecialCharTok{==}\StringTok{"RI"}\NormalTok{] }\OtherTok{\textless{}{-}} \StringTok{"Northeast"}
\NormalTok{beach3}\SpecialCharTok{$}\NormalTok{Region[beach3}\SpecialCharTok{$}\NormalTok{State}\SpecialCharTok{==}\StringTok{"MA"}\NormalTok{] }\OtherTok{\textless{}{-}} \StringTok{"Northeast"}
\NormalTok{beach3}\SpecialCharTok{$}\NormalTok{Region[beach3}\SpecialCharTok{$}\NormalTok{State}\SpecialCharTok{==}\StringTok{"NH"}\NormalTok{] }\OtherTok{\textless{}{-}} \StringTok{"Northeast"}
\NormalTok{beach3}\SpecialCharTok{$}\NormalTok{Region[beach3}\SpecialCharTok{$}\NormalTok{State}\SpecialCharTok{==}\StringTok{"ME"}\NormalTok{] }\OtherTok{\textless{}{-}} \StringTok{"Northeast"}
\NormalTok{beach3}\SpecialCharTok{$}\NormalTok{Region[beach3}\SpecialCharTok{$}\NormalTok{State}\SpecialCharTok{==}\StringTok{"MP"}\NormalTok{] }\OtherTok{\textless{}{-}} \StringTok{"Hawaii{-}Pacific Islands"}
\NormalTok{beach3}\SpecialCharTok{$}\NormalTok{Region[beach3}\SpecialCharTok{$}\NormalTok{State}\SpecialCharTok{==}\StringTok{"GU"}\NormalTok{] }\OtherTok{\textless{}{-}} \StringTok{"Hawaii{-}Pacific Islands"}
\NormalTok{beach3}\SpecialCharTok{$}\NormalTok{Region[beach3}\SpecialCharTok{$}\NormalTok{State}\SpecialCharTok{==}\StringTok{"MK"}\NormalTok{] }\OtherTok{\textless{}{-}} \StringTok{"California Current"}
\NormalTok{beach3}\SpecialCharTok{$}\NormalTok{Region[beach3}\SpecialCharTok{$}\NormalTok{State}\SpecialCharTok{==}\StringTok{"PR"}\NormalTok{] }\OtherTok{\textless{}{-}} \StringTok{"Caribbean"}

\CommentTok{\#combine}
\NormalTok{beach4 }\OtherTok{\textless{}{-}}\FunctionTok{with}\NormalTok{(beach3,}\FunctionTok{aggregate}\NormalTok{(}\FunctionTok{list}\NormalTok{(}\AttributeTok{days=}\NormalTok{beach3}\SpecialCharTok{$}\StringTok{\textasciigrave{}}\AttributeTok{ActionDuration Days}\StringTok{\textasciigrave{}}\NormalTok{), }\AttributeTok{by=}\FunctionTok{list}\NormalTok{(}\AttributeTok{region=}\NormalTok{beach3}\SpecialCharTok{$}\NormalTok{Region,}\AttributeTok{year=}\NormalTok{beach3}\SpecialCharTok{$}\NormalTok{Year), sum))}

\CommentTok{\#reorder}
\NormalTok{beachwide }\OtherTok{\textless{}{-}} \FunctionTok{spread}\NormalTok{(beach4, region, days)}
\NormalTok{beachwide}

\FunctionTok{write.csv}\NormalTok{(beachwide,}\AttributeTok{file=}\StringTok{"C:/Users/willem.klajbor/Documents/DataAssembly/output/beach.csv"}\NormalTok{)}
\end{Highlighting}
\end{Shaded}

\hypertarget{marine-species-distribution}{%
\chapter{Marine Species Distribution}\label{marine-species-distribution}}

\hypertarget{data-23}{%
\section{Data}\label{data-23}}

Under Construction

\hypertarget{methods-23}{%
\section{Methods}\label{methods-23}}

Under construction

  \bibliography{book.bib,packages.bib}

\end{document}
